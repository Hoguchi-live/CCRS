\documentclass[../main.tex]{subfilesubs}
\begin{document}

\subsection{Global architecture of the code}
Here we introduce the relations between the different files and folders of the code.
At the top level there are three folders: \texttt{example/}, \texttt{optimization/} and \texttt{src/}

Folder \texttt{example/} holds a working instance of a key-exchange using the algorithm.
A \textit{bash} script is provided to compile the example.
This script accepts two flags:
	\begin{itemize}
		\item -DVERBOSE which will print some live information on the key-exchange; and
		\item -DTIMING which will output timings for the isogeny steps in \textit{json} format.
	\end{itemize}
This second flag is used to perform optimization on the $l$-primes bounds.

The \texttt{optimization/} folder contains two pyhon scripts.
The \texttt{optimize.py} script loads a \textit{timings.json} file located in subfolder \texttt{files} and created using the -DTIMING flag.
It uses the \textit{GEKKO} optimizer to find the best combination of bounds on the steps for each $l$-primes.
This optimizes the \textit{timing} of the global algorithm while keeping the \textit{keyspace} size above $256$ bits.
The results can then be found in \texttt{files/}.
A plotting script is also available to graph the speed of steps against $l$.
See the third section on results for more details.

Last and most importantly, the \texttt{src/} folder contains all the necessary functions and structures implementing the CRS protocol.
It is once again divided in four subfolders.

In \texttt{EllipticCurves/} one will find all the functions and structures related to curve manipulation.
The \texttt{models.h} file holds the type definition of elliptic curves in different forms and their corresponding points.
All functions related to memory management are located in the \texttt{memory.c/h} files.
The exact structures holding the elliptic curves are discussed in more details in the next section.
The \texttt{arithmetic.c/h} files contain all the mathematical functions directly related to the curves.
For instance the \textit{Montgomery Ladder}, the extraction of random rational torsion points and the model transformation functions are located there.
The \texttt{auxiliary.c/h} files only serve as shortcuts to unavailable FLINT functions.
The \texttt{pretty\_print.c/h} files define convenient debugging functions mainly used to check on the key-exchange.

Folder \texttt{Isogeny/} holds the two algorithms used to compute isogenies between elliptic curves.
The \texttt{radical.c/h} files implement radical isogenies for primes $3$, $5$ and $7$.
Files \texttt{velu.c/h} account for the $\sqrt{}$-Velu algorithm described further in the paper.
Finally \texttt{walk.c/h} holds the two handle functions used to walk in the isogeny graph using the aforementioned algorithms.

The \texttt{Polynomial/} folder contains the binary-tree based multi-evaluation algorithm in \texttt{multieval.c/h}.
The binary-tree structure and functions themselves are defined in file \texttt{binary\_trees.c/h}.
Alongside with \texttt{roots.c/h} hosting a bruteforce square root extraction algorithm.
This was necessary as no root-extraction methods are available for large fields in the FLINT library.

Finally the \texttt{Exchange/} folder hosts everything related to the global structure of the CRS protocol.
The global environment setup is handled in \texttt{setup.c/h} while secret key generation happens in \texttt{keygen.c/h}.
Applying a key to an elliptic curve is done through the single function located in the \texttt{dh.c} file.
Auxiliary files \texttt{info.c/h} are there to print live information during the exchange  and to get timings for optimization.


\subsection{C structures and memory management}
This section will first introduce the different custom types used in the code to handle context and mathematical objects.
We have made the choice to clearly represent in C the objects we use.
To that extent we defined a number of structures representing elliptic curves, their points and other less obvious data in memory.
We follow the GMP/FLINT standard nomenclature identifying types with a "\_t".

The file \texttt{EllipticCurves/models.h} defines the three types of elliptic curve models we use.
These \textit{structs} named respectively \tF{SW_curve_t}, \tF{MG_curve_t} and \tF{TN_curve_t} stand for Short Weierstrass, Montgomery and Tate normal curves.
Inside these \textit{struct} lies a \tF{const fq_ctx_t *} pointer to the field of definition of the curve.
A pointer is used in order to not re-instantiate the whole field structure for every curves.
The multiple \tF{fq_t} parameters respectively named $a$, $b$; $A$, $B$; $b$, $c$ are the respective models' parameters.
The \tF{fmpz_t} parameter $l$ in the definition of type \tF{TN_curve_t} is used to keep track of the order of the point $(0, 0)$ on the curve.
These all strictly follow the definitions found in the first part of this paper.

The \textit{structs} named respectively \tF{SW_point_t} and \tF{MG_point_t} represent points on the respective curves.
These \textit{structs} contain the coordinate of a point on a curve as \tF{fq_t} elements.
Said curve is linked to the point via a pointer of the respective type.

In file \texttt{Polynomials/binary\_tree.h} we define a binary-tree structure used in the fast-multipoint polynomial evaluation.
The \textit{struct} type \tF{fq_poly_btree_t} holds a \tF{fq_ctx_t *}  pointer to keep track of the field of definition of its elements.
It also holds a pointer to the head (or root) cell of the tree.
This decision to distinguish the whole tree from its cells stems from fear of memory management issues.

The cells are represented by the \tF{fq_poly_bcell_t} type.
This structure also holds a pointer to the field $k$ we are working with.
Most importantly, it holds a \tF{fq_poly_t} which is is a polynomial over $k$.
The remaining fields are \tF{fq_poly_bcell_t} pointers to the left and right child of this cell.
Keyword \textit{struct fq\_poly\_bcell\_t} is used to recursively define these childs in the type definition.
Both should be initialized to \textit{NULL} to represent a leaf of the tree.

File \texttt{Exchange/setup.h} holds the \tF{lprime_t} structure representing $l$-primes.
A specific structure has been assigned for this task because of the numerous parameters surrounding $l$-primes.
These consists only of an \tF{fmpz_t} representing $l$ and multiple \textit{unsigned int}.
Integers \textit{lbound} and \textit{hbound} account for the bounds on the number of steps allowed for the walk on the $l$-isogeny graph.
It turns out that both upper and lower bounds are equal since we managed to avoid root extraction on the backward direction by using the quadratic twist.
This structure also holds the extension degree $r$ corresponding to the Frobenius eigenvalue's order mod $l$.
Finally it also contains a uint serving as a boolean to allow or disallow backward walking.
This is necessary as some $l$-primes have one eigenvalue of low order but another one of unusably high order.

In this same file, the global configuration type \tF{cfg_t} is defined.
This type holds the public starting curve as an \tF{MG_curve_t}.
Furthermore, it contains the number of $l$-primes to use in the protocol as well as the \tF{lprime_t} array holding them.
The base field and its extensions up to degree $9$  are held in an \tF{fq_ctx_t} array to avoid re-instantiation.
Finally an unsigned integer defined as a seed for the random key generation.
This last step allow for debugging but also for timing tests.

In file \texttt{Exchange/keygen.h} one can find the definition of type \tF{key__t} representing a secret CRS key.
This twisted nomenclature avoids the already defined \tF{key_t} type while remaining as consistent as possible.
A key holds the number of primes used in the protocol, the list of \tF{lprime_t} used and an \tF{fmpz_t} list holding the bounds on the steps to take for each prime.
Note that these lists should all be ordered from the smallest working extension degree to the largest.
The order of the $l$-primes themselves within the same extension degree does not matter.
This is done to avoid downgrading the base field.
This way we only upgrade (embed) the base field once per degree (refer to the \tF{MG_curve_update_field} function  in \texttt{Exchange/dh.h} for details).

Memory management was the first concern of this project and as such was dealt with very rapidly.
Since we are using GMP through FLINT, we decided to use the same \textit{declare, init, set, clear} paradigme.
That way almost all our functions start by a declaration of the variables used.
Followed by an initialization, often using an \tF{fq_ctx_t} field.
Then the data is set and used accordingly then finally the memory is cleard.
This is respected by all of our custom types.
The corresponding management functions are named following the nomenclature \tF{*_init}, \tF{*_set} and \tF{*_clear} where \tF{*} is the type in question.
Some additional practical functions like \tF{MG_curve_set_si} or \tF{fq_poly_bcell_set_right} allow for shortcuts.
Still they were inspired by the FLINT ecosystem.

Most of our custom initialization and clearing functions are just wrappers for FLINT functions.
This offers a safer and more comfortable handling of the many objects in the code.
However there are multiple instances where manual memory allocation was favorable.
This is the case for instance with binary trees, $l$-primes, keys and config types.
The main reason was for allocating memory for arrays of custom data and/or not having to pre-declare objects.
Sometimes both approaches were used.
This can be seen in \texttt{Exchange/keygen.c} with \tF{key_init} initializing a declared key and \tF{key_init_} returning a pointer to an initalized key.

\subsection{Montgomery curve arithmetic}

\newcommand{\x}{\textbf{x}}
\newcommand{\funct}[1]{\texttt{\detokenize{#1}}}
We chose to implement  the Montgomery curve arithmetic as described by Craig Costello and Benjamin Smith \cite{}.
It is called x-only arithmetic because any finite projective point $P = (X:Y:Z)$ of $E$ can be mapped to a point $\x(P)=(X:Z)=(X/Z:1)\in\PP^1$ if $Z\neq0$ or to $\x(P)=(1:0)$ if $Z=0$.
Any operation on a finite point $P$ is done after discarding its $Y$ coordinate. This coordinate can be recovered later if needed.

\begin{lemma}Assume $X_{P-Q}\neq0$.

	 If $P\neq Q$ then  \begin{equation}\left\{\begin{array}{l}
	 	X_{P+Q} = Z_{P-Q}[(X_P-Z_P)(X_Q+Z_Q) + (X_P+Z_P)(X_Q-Z_Q)]^2\\
	 	Z_{P+Q} = X_{P-Q}[(X_P-Z_P)(X_Q+Z_Q) - (X_P+Z_P)(X_Q-Z_Q)]^2
	 \end{array}\right.\end{equation}

 	If $P=Q$ then \begin{equation}\label{xdbleq}\left\{\begin{array}{l}
 		X_{[2]P}=(X_P + Z_P)^2(X_P - Z_P)^2 \\
 		Z_{[2]P}=(4 X_P Z_P)[(X_P - Z_P)^2 + \frac{A+2}{4}(4 X_P Z_P)]
 	\end{array}\right.\end{equation}
\end{lemma}

Using this lemma, we provide three generic pseudo-operations on Montgomery curve points that constitute the framework for x-only arithmetic in our implementation : addition, doubling and scalar multiplication.

These algorithms are called pseudo-operations because they operate over the quotient set of the curve $E$ by the partition
$\lbrace\lbrace P, -P\rbrace, P\in E\rbrace$. e.g. every point is identified with its opposite. This identification is implied in the following paragraphs.

\paragraph{xADD}The pseudo-addition or differential addition \funct{MG_xADD} returns $P+Q$ given $P,Q\in E$ and their difference $P-Q$ using the formula.
Our implementation of xADD uses 4 multiplications, 2 squaring,  3 additions and 3 subtractions in $\FF_p$, for a total of  12 base operations over $\FF_p$.
However the point addition xADD requires to first compute the difference $P-Q$, therefore it cannot be used as a generic point addition algorithm but can only be used in the specific context of a differential addition chain.

\paragraph{xDBL}The pseudo-doubling \funct{MG_xDBL} computes $[2]P$ from input $P$.  The cost is 3 additions, 2 squaring, 2 subtractions, 3 multiplications and 1 division in $\FF_p$, for a total of 11 operations. When calling xDBL multiple times for points on the same curve, one could cache or precompute the value $\frac{A+2}{4}$ from equation \eqref{xdbleq} and therefore shave off 1 addition and 1 division from the cost of every subsequent calls to xDBL.

\paragraph{Montgomery Ladder} \funct{MG_ladder} computes the point multiplication $[k]P$ calling xADD and xDBL. Write $k=\sum_{i=0}^{l-1}k_i2^i$. The Montgomery Ladder follows a differential addition chain of length $2l-1$ wherein the difference of consecutive terms is constant. The execution of \funct{MG_ladder} requires $l$ calls to xDBL and $l-1$ calls to xADD.
\begin{minted}[frame=lines, linenos, ]{c}
P0 = P
P1 = xDBL(P)
for(int i=l-2; i>=0, i--) {
	if(k[i]==0){
		P1 = xADD(P0, P1, P)
		P0 = xDBL(P0)
	}
	else{
		P0 = xADD(P0, P1, P)
		P1 = xDBL(P1)
	}
}
return P0
\end{minted}
In the specific case of the Montgomery Ladder
Note that the Montgomery ladder can be implemented using a conditional constant-time swap to prevent branching or timing attacks.
\end{document}
