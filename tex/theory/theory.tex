\documentclass[../main/main.tex]{subfiles}
\begin{document}

In this section we introduce the necessary mathematical material surrounding the CRS protocol.
It is assumed the reader has some familiarities with elliptic curves and their arithmetic.
Still, we recall the fundamentals in order to underline some important aspect of the algorithms.
We fix once and for all a prime number $p$.
Unless indicated otherwise we assume throughout the section that curves are defined over $\FF_p$, the field with $p$ elements.

%The \textit{Montgomery} and \textit{Tate normal} forms are introduced

\subsection{Elliptic curves and models}
Recall the standard definition of an elliptic curve over $k$:
\begin{definition}
	An elliptic curve $E$ over a field $k$ is a smooth curve given by a (dehomogenized) polynomial $F\in k[x,y]$ of the form
	\[
		F = y^2 + a_1xy + a_3y - (x^3 +a_2 x^2 + a_4 x + a_6).
	\]
\end{definition}
This presentation of the curve is usually called the \textit{Weierstrass} form.
It is said to be in \textit{short Weierstrass} form if the coefficients $a_1$, $a_2$ and $a_3$ are null.
The following proposition allow one to only consider short Weierstrass equations except for some fields.
\begin{proposition}
	An elliptic curve $E$ over a field $k$ of caracteristic different from 2 and 3 is isomorphic to an elliptic curve in short Weierstrass form.
\end{proposition}
\begin{proof}
	See Silverman III.1. \cite{}
\end{proof}
The (short-)Weierstrass models admit explicit formulas that reflect the group law of the elliptic curve.
It turns out that other models have better, faster properties when it comes to group operations.
The following \textit{Montgomery} model has one of the fastest doubling operation of all.
It also has a fast scalar multiplication which we talk about in the next section.
\begin{definition}
	A \textit{Montgomery curve} or an elliptic curve in \textit{Montgomery} form over a field $k$ is an elliptic curve $E_{A, B}$ given by a polynomial $F\in k[x, y]$ of the form
	\[
		F = By^2 - x(x^2+Ax + 1)
	\]
	with $A, B\in k$ satisfying $b\neq 0$ and $A^2\neq 4$.
\end{definition}
All Mongomery curves admit a short Weierstrass model via the variable change $x := x/B$ and $y := y/B$.
The following proposition show which short Weierstrass curves admit a Montgomery model.
\begin{proposition}
	The elliptic curve $y^2 = x^3 + ax + b$ over a field $k$ admits a Montgomery model if
	\begin{itemize}
		\item The polynomial $X^3 + aX + b$ has a root $w$ in $k$
		\item $3w^2 + a$ is a quadratic residue in $k$.
	\end{itemize}
\end{proposition}
\begin{proof}
	Under the above conditions, set $u = (3w^2+a)^{-\frac{1}{2}}$.
	The (admissible) variable change $x:= u(x-w)$, $y:= uy$ show that the curve has Montgomery model $E_{A, B}$ with $A = 3wu$ and $B=u$.
\end{proof}

Associated to an elliptic curve $E$ is the fundamental quantity called $j$\textit{-invariant}.
For a Mongomery curve $\MG$ over a field $k$ of caracteristic different from $2$ and $3$ one has
\[
	j = \frac{256(A^3-3)^3}{A^2 - 4}.
\]

Elliptic curves can be shown to be isomorphic if and only if they have the same $j$-invariant.
The crucial thing to remember is that this classification only happens over the algebraic closure $\overbar{k}$ of the base field $k$.
Curves can be isomorphic over $\overbar{k}$ while being non-isomorphic over $k$ or a finite extension of $k$.
In fact, the $B$ parameter in the definition of a Mongomery curve account for two isomorphism classes: take $A, B, B'\in k$ and consider the curve $\MG$ and $E_{A, B'}$.
One easily show that they are isomorphic at best over $k(\sqrt{B/B'})$.
Hence they are isomorphic over $k$ if and only if $B/B'$ is a quadratic residue.
We call $E_{A, B'}$ a \textit{quadratic twist} of $\MG$.

The next and last model of elliptic curve is called the \textit{Tate normal} form.
It will be used when dealing with radical isogenies.
\begin{definition}
	An Elliptic curve in \textit{Tate normal} form over a field $k$ is an elliptic curve $E$ given by a polynomial $F\in k[x, y]$ of the form
	\[
		F = y^2 + (1-c)xy - by - x^2(x-b)
	\]
	on which the point $(0,0)$ has order at least $4$.
	We also say $E$ is in Tate normal form when given by
	\[
		F = y^2 + (1-c)xy - by - x^3
	\]
	where $(0,0)$ is of order $3$.
\end{definition}
The following lemma and its proof show how to transform a Mongomery curve along with a point $P$ of order $N\geq 4$ into Tate normal form.
We demonstrate the lemma with a general Weierstrass curve but only the Montgomery case with $B=1$ interests us in the implementation.
It is only a matter of setting some $a$-coefficients to $0$ in the formulas.
%The same procedure can be followed for points of order 3 and is available in the source code.
\begin{lemma}
	Let $E$ be an elliptic curve over a field $k$ and let $P=(x, y)\in E$ be a point of order $N\geq 3$.
	There exist a Tate normal model for $E$ such that $P$ is sent to $(0,0)$..
\end{lemma}
\begin{proof}
	We consider $E$ in general Weierstrass form
	\[
		Y^2 + a_1XY + a_3Y = X^3 +a_2 X^2 + a_4 X + a_6.
	\]
	A translation from $P$ to $(0, 0)$ allows us to remove $a_6$ and write the curve as
	\[
		Y^2 + a_1XY + b_3Y = X^3 +b_2 X^2 + b_4 X.
	\]
	where $b_2 = 3x+a_1$, $b_3 = 2y + a_1x + a_3$ and $b_4 = 3x^2 +2a_1x+a_4$.
	The coefficient $b_3$ is non-zero as $P$ does not have order $2$.
	This can be read on the duplication formula.
	Let us remove the $b_4$ coefficient by the (admissible) change of variable $Y :=Y + b_4 /b_3  X$.
	We get the representation
	\[
		Y^2 + c_1XY + b_3Y = X^3 +c_2 X^2.
	\]
	where  $c_1 = 2 b_4 / b_3 + a_1$ and $c_2 = b_2 - a_1b_4/b_3$.

	Now introduce two free scaling variables $\alpha, \beta\in k$.
	The scaling $X = \alpha X$, $Y = \beta Y$ gives a model that is Tate normal if
	\begin{align*}
		{\alpha}^3 &= 	{\beta}^2 \\
		\beta b_3 &= c_2 {\alpha}^2.
	\end{align*}
	Following the standard negation formula and $N=3$ being equivalent to $2P = -P$ we see that $c_2 = 0$ is also equivalent to $N=3$.
	Assuming that $N\geq 4$, one find $\alpha = ({b_3}/c_2)^2$ and $\alpha = c_2/b_3$.
	Thus setting $b = -c_2(b_3/c_2)^2$ and $c = 1-c_1(c_2/b_3)^2$ gives the Tate normal form.
	If  $N=3$, the curve is already in Tate normal form.
\end{proof}
Notice however that in the case $N=3$, the coefficient $b_3$ should be computed to get $b$ and $c$.
This in turns involves knowing the value of $y$.
As we will see with $x$-only arithmetic, over Mongomery curves it is better to use our free system in $\alpha$ and $\beta$.
Setting $\alpha = b_3^3$ and $\beta = b_3^2$ we find a presentation which only involves $y^2$.
In the end we find $b = -1/b_3^2$ and $c = 1 - 2b_4 / b_3^2$.
This will avoid us extracting a square root of $x^3 + ax^2 + 1$ to get a $y$ value.






\end{document}
