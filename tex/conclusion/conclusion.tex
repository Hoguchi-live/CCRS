\documentclass[../main.tex]{subfiles}
\begin{document}
This project served as an interesting way to learn the basics of the C language applied to mathematics.
As always the algorithms were straightforward on paper but much less so were the actual implementations.
Having full control over memory allocation and de-allocation was complicated but paid off in the end.

Further optimizations of the protocol could now be pursued.
Better radical formuls for primes $11$ and $13$ could greatly enhance the timing results as these are two orders of magnitude faster than the worst $l$-primes.
Parallelization of the multi-evaluation algorithm through multi-threading could speed up the $\sqrt{}$-Velu algorithm by quite a bit.
This could also be inverstigated on a GPU.

At the end of the day there is no denying that this protocol is still extremely slow compared to its SIDH variants.
However the great leaps from one implementation to the next show that there is plenty of room for new techniques and optimizations.

\end{document}
