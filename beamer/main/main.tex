
\documentclass{beamer}
\usetheme{default}


\title{Implementation of the isogeny-based key-exchange protocol CRS}
\author{Hugo Nartz, Cl\'ement Jacquot}
\date{February 16, 2022}


\AtBeginSubsection[]
{
  \begin{frame}<beamer>{Outline}
    \tableofcontents[currentsection,currentsubsection]
  \end{frame}
}

\begin{document}

\begin{frame}
  \titlepage
\end{frame}

\begin{frame}{}
  \tableofcontents
\end{frame}

\section{Définitions et permier résultat}

\begin{frame}{Fraction rationelle}
  \begin{itemize}
  \item {
    $ n \in \mathbb{N}$
  }
  \item {
    $X_0, X_1, \dots, X_n$ des variables
  }
  \item{
    $F_0 = X_0$
    }
  \item{
    $F_{n+1}(X_0,\cdots, X_{n+1} ) = F_n(X_0,\cdots,X_n +\frac{1}{X_{n+1} } ) $}
    \item{Notation: $F_n = [X_0,\cdots, X_n]$}
  \end{itemize}
\end{frame}

\begin{frame}{Réduite et quotients}
\begin{itemize}
    \item{}
\end{itemize}
\end{frame}

\begin{frame}{Proposition}
\begin{itemize}
    \item{Il existe deux suites $(P_n)$ et $(Q_n)$ de polynomes tels que:}
    \item{$P_n$ et $Q_n$ ne dépendent que de $X_0,\cdots,X_n$}
    \item{$P_0 = X_0, P_1 = X_0 X_1+1$ et $Q_0 = 1, Q_1 = X_1$}
    \item{$\forall n \geq 2: P_n = X_nP_{n-1}+P_{n-2}$ et $Q_n = X_n Q_{n-1}+Q_{n-2}$}
    %deg ?
\end{itemize}
\end{frame}

\begin{frame}{Théorème}
\begin{itemize}
    \item{$\forall n \geq 0$}
    \item{$F_n = [X_0,\cdots, X_n] = \frac{P_n}{Q_n}$}
\end{itemize}
\end{frame}

\section{Algorithme des fractions continues}
\begin{frame}{Algorithme}
\begin{itemize}
    \item{$\theta \in \mathbb{R}$}
    \item{$a_0 = [\theta]$}
    \item{Si $\theta\in \mathbb{Z}$ alors $\theta = a_0$ fin}
    \item{Sinon $\theta-a_0 \in ]0,1[$ et il existe $\theta_1$ tel que $\theta = a_0 + \frac{1}{\theta_1}$, on réitère le processur pour $\theta = \theta_1$ et $a_1=[\theta_1]$}
    \item{L'algorithme termine si et seulement si il existe $n$ tel que $\theta_n = [a_n]$}
\end{itemize}
\end{frame}




\end{document}
